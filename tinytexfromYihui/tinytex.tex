% This is a Pandoc template created from http://tug.org/TUGboat/sampleart.ltx
\documentclass{ltugboat}

\usepackage{microtype}
\usepackage{graphicx}
\usepackage{ifpdf}
\ifpdf
\usepackage[breaklinks,hidelinks,pdfa]{hyperref}
\else
\usepackage{url}
\fi

\def\macOS{mac\acro{OS}}
\def\GB{\acro{GB}}
\def\tinytex{\texttt{tinytex}}

\title{TinyTeX: A lightweight, cross-platform, and easy-to-maintain LaTeX
distribution based on TeX Live}

\author{Yihui Xie}
\address{RStudio, Inc.}
\netaddress{xie@yihui.name}
\personalURL{https://yihui.name}

\begin{document}

\maketitle

\begin{abstract}
As a \LaTeX{} user for 15 years, I have suffered from two problems related
to the installation of \LaTeX{} and maintenance of packages: 1) The full
versions of common \LaTeX{} distributions are often too big, whereas the
smaller basic versions often lack packages that I frequently use; 2) It
is tedious to figure out which missing packages to install by reading
the error log from the \LaTeX{} compilation. TinyTeX
(\url{yihui.name/tinytex/}) is my attempt to address these
problems. The basic version of TinyTeX is relatively small (150\acro{MB} on
Linux/\macOS when installed), and you only install additional packages
if/when you actually need them. Further, if you are an R user, the
installation of missing packages can be automatic when you compile \LaTeX{}
or R Markdown documents through the R package \tinytex{}.
\end{abstract}

\hypertarget{motivation}{%
\section{Motivation}\label{motivation}}

If you do not want to be bothered by \LaTeX{} errors that tell you certain
class or style files are missing, one way to go is to install the full
version of the \LaTeX{} distribution, which typically contains the vast
majority of packages on CTAN. Take \TeX{} Live for example. The size of its
full version is 4 to 5\GB. Yes, I do hear the argument that the hard disk
storage is fairly cheap today. Why should this 5\GB bother us at all? The
problems are:

\begin{itemize}
\item
  It can take a long time to download, although we usually do this only
  once a year. However, if you use a cloud service for continuous
  integration or testing (e.g., \href{https://travis-ci.org}{Travis \acro{CI}})
  of your software package that depends on \LaTeX{}, this can be worse,
  because each time you update your software (e.g., though a \acro{GIT}
  commit), the virtual machine or cloud container downloads 5\GB again.
\item
  It contains a lot of \LaTeX{} packages that an average user does not
  need. I do not know if I'm a representative user, but for the more
  than 5600 packages on CTAN, I routinely use less than 100 of them. In
  other words, I'm just wasting my disk space with more than 98\% of the
  packages.
\item
  It takes much longer to update packages if you choose to update all
  via \texttt{tlmgr\ update\ -\/-all} (and you will be installing newer
  versions of packages that you do not need, too).
\end{itemize}

Without installing the full version, you may be confused when compiling
a document and a needed packages is not installed. The report at
\url{github.com/rstudio/rmarkdown/issues/39} is a good example
to show how users can be confused. The main reason for the confusion is
that an error message like below does not tell you how you could resolve
the issue (i.e., which package to install and how to install it):

\begin{verbatim}[\small]
! Error: File `framed.sty' not found.

Type X to quit or <RETURN> to proceed,
or enter new name. (Default extension: sty)

Enter file name: 
! Emergency stop.
<read *> 
\end{verbatim}

Even worse, \TeX{} Live can be different on different platforms. For
example, if you use a Linux distribution's packaging of \TeX{} Live,
typically you cannot just install the (system) package named
\texttt{framed} even if you know \texttt{framed.sty} is from the (\LaTeX{})
package \texttt{framed}, because \TeX{} Live is often made available by
distributions as \emph{collections} of \LaTeX{} packages, so you have to
figure out which system package contains the \LaTeX{} package
\texttt{framed}. Is it \texttt{texlive-framed}, or
\texttt{texlive-latex-extra}? On another front, if you use MacTeX (which
is essentially \TeX{} Live) on \macOS, you would usually run
\texttt{sudo\ tlmgr\ install\ framed}, hence type your password every
time you install a package.

Then the next year when a new version of \TeX{} Live is released, you may
have to go through the same pain again: either waste your disk space, or
waste your time. One interesting thing I noticed from \macOS users was
that many of them did not realize that each version of MacTeX was
installed to a different directory. For example, the 2018 version is
installed under \texttt{/usr/local/texlive/2018}, and the 2017 version
is under \texttt{/usr/local/texlive/2017}. When they started to try
TinyTeX (which recommended that they remove their existing \LaTeX{}
distribution), they had realized for the first time that there were five
full versions of \TeX{} Live on their computer, and they were very happy to
suddenly regain more than 20\GB of disk space.

I wished there were a \LaTeX{} distribution that contains only packages I
actually need, does not require \texttt{sudo} to install packages, and
is not controlled by system package managers like \texttt{apt} or
\texttt{yum}. I wished there were only one way to manage \LaTeX{} packages
on different platforms. Fortunately, the answer is still \TeX{} Live, just
with a few tricks.

\hypertarget{the-infraonly-scheme-and-the-portable-mode-to-the-rescue}{%
\section{\texorpdfstring{The \texttt{infraonly} scheme and the portable
mode to the
rescue!}{The infraonly scheme and the portable mode to the rescue!}}\label{the-infraonly-scheme-and-the-portable-mode-to-the-rescue}}

There are three possible ways to cut down the size of \TeX{} Live:

\begin{enumerate}
\def\labelenumi{\arabic{enumi}.}
\item
  Only install the packages you need.
\item
  Do not install the package source.
\item
  Do not install the package documentation.
\end{enumerate}

The first way can be achieved by installing a minimal scheme of \TeX{} Live
first, which includes its package manager \texttt{tlmgr}, and then
install other packages via \texttt{tlmgr\ install}. The minimal scheme
is named \texttt{scheme-infraonly}, and it is only about 10\acro{MB}.

The second and third ways can be specified through installation options,
which I will mention soon. The package documentation contributes a
considerable amount to the total size of a \TeX{} Live installation.
However, I have to admit I rarely read them, and I do not even know
where these documentation files are on my computer. When I have a
question, I will almost surely end up in a certain post on
\url{tex.stackexchange.com}, and find a solution there. It is
even rarer for me to read the package source files, since I am not a
\LaTeX{} expert, nor am I interested in becoming an expert.

With the network installer of \TeX{} Live
(\url{tug.org/texlive/acquire-netinstall.html}), we can put the
above pieces together, and automate the installation through an
``installation profile'' file. Below is the one that I used for TinyTeX
(named \texttt{tinytex.profile}):

\begin{verbatim}[\small]
selected_scheme scheme-infraonly

TEXDIR ./
TEXMFSYSCONFIG ./texmf-config
TEXMFLOCAL ./texmf-local
TEXMFSYSVAR ./texmf-var

option_doc 0
option_src 0
option_autobackup 0

portable 1
\end{verbatim}

The installation is done through

\begin{verbatim}[\small]
./install-tl -profile=tinytex.profile
\end{verbatim}

where \texttt{install-tl} is extracted from the Net installer (use
\texttt{install-tl-windows.bat} on Windows). The full source of the
installation scripts can be found on Github at
\url{github.com/yihui/tinytex/tree/master/tools}. To install
TinyTeX on *nix, run \texttt{install-unx.sh}; to install it on Windows,
run \texttt{install-windows.bat}.

I set the \texttt{portable} option to \texttt{1} above, which means the
installation directory will be portable. You can move it anywhere in
your system, as long as you know how to handle the \texttt{PATH}
variable, or call the executables (e.g., \texttt{tlmgr} or
\texttt{pdflatex}) with their full paths. By default, the installation
scripts of TinyTeX will try to add \TeX{} Live's bin path to the
environment variable \texttt{PATH}, or create symlinks to a path that is
in \texttt{PATH} (e.g., \texttt{/usr/local/bin} on \macOS and
\texttt{\$HOME/bin} on Linux).

A portable installation without admin privileges also means anyone can
install and use \TeX{} Live on any platforms supported by \TeX{} Live. You can
also install a copy to a \acro{USB} device and use it from there. Users inside
an institute no longer need to ask for \acro{IT} help with managing \LaTeX{}
packages because of the powerful and useful \texttt{tlmgr}. With
TinyTeX, \texttt{tlmgr} is the one and only way to manage packages
directly, and you will not need \texttt{sudo}, \texttt{apt}, or
\texttt{yum}.

\hypertarget{the-r-package-tinytex-install-missing-latex-packages-on-the-fly}{%
\section{\texorpdfstring{The R package \tinytex{}: install missing
\LaTeX{} packages
on-the-fly}{The R package tinytex: install missing \LaTeX{} packages on-the-fly}}\label{the-r-package-tinytex-install-missing-latex-packages-on-the-fly}}

Now I only have one last wish for \TeX{} Live: I wish it could install
missing packages on-the-fly like MiKTeX when compiling documents. I do
not know how MiKTeX implemented it. I'm primarily an R \cite{R-base}
package developer. I do not know much about the \TeX{} language or Perl. I
know how to search for the package that contains a certain style or
class file and install it, e.g.,

\begin{verbatim}[\small]
$ tlmgr search --global --file "/times.sty"
psnfss:
      texmf-dist/tex/latex/psnfss/times.sty
...
$ tlmgr install psnfss
\end{verbatim}

I had done this too many times in the past, and thought it might be
possible to automate it. I made an attempt in the R package
\tinytex{} \cite{R-tinytex}. I guess \LaTeX{} experts may frown upon
my implementation, but it was the best I could do, given my limited
capabilities and knowledge in \LaTeX{}.

Basically, I try to compile a \LaTeX{} document via an engine like
\texttt{pdflatex} or \texttt{xelatex}, with arguments
\texttt{-halt-on-error} and \texttt{-interaction=batchmode}. If the exit
status is non-zero, I will parse the error log and find the error
messages. If I made any contribution at all, it would be the following
possible error messages that I collected in about a year:

\begin{verbatim}[\small]
! \LaTeX{} Error: File `framed.sty' not found.
/usr/local/bin/mktexpk: line 123: mf:
  command not found
! Font U/psy/m/n/10=psyr at 10.0pt not
  loadable: Metric (TFM) file not found
!pdfTeX error: /usr/local/bin/pdflatex
  (file tcrm0700): Font tcrm0700 at 600 not found
! The font "FandolSong-Regular" cannot be found.
! Package babel Error: Unknown option
  `ngerman'. Either you misspelled it
(babel)                or the language
  definition file ngerman.ldf was not found.
!pdfTeX error: pdflatex (file 8r.enc):
  cannot open encoding file for reading
! CTeX fontset `fandol' is unavailable in
  current mode
Package widetext error: Install the
  flushend package which is a part of sttools
Package biblatex Info: ... file
  'trad-abbrv.bbx' not found
! Package pdftex.def Error: File
  `logo-mdpi-eps-converted-to.pdf' not found
\end{verbatim}

In the R package \tinytex{}, I try to obtain the names of the
missing files or fonts or commands (e.g., \texttt{framed.sty},
\texttt{mf}, \texttt{tcrm0700}), run \texttt{tlmgr\ search} to obtain
the package name, and \texttt{tlmgr\ install} the package if possible.

The thing that \TeX{} Live experts may frown upon is that since I do not
know all possible missing packages beforehand, I will just keep trying
to compile the document, find the missing packages, and install them. In
other words, I do not know if there is a missing package unless I
actually compile the document and hit an error. If a document contains
\(n\) missing packages, it may be recompiled \(n\) times.

On the bright side, this only needs to be done at most once for a
document, so even if it is slow for the first time, the compilation will
be much faster next time because all necessary packages have been
installed. The process is also automatic (by default), so all you need
to do is wait for a short moment. This feature is turned on for R
Markdown \cite{R-rmarkdown} users, which means if the user's \LaTeX{}
distribution is TinyTeX, they will almost never run into the issue of
missing packages when compiling R Markdown to PDF, and the
``easy-to-maintain'' TinyTeX should not need maintenance at all. As a
matter of fact, this article
\href{https://github.com/yihui/tinytex/blob/master/TUGboat/tinytex.Rmd}{was
written in R Markdown}, and the first time I compiled it, the
\texttt{tugboat} package was automatically installed:

\begin{verbatim}[\small]
tlmgr search --file --global /tugboat.cls
tlmgr install tugboat
...
[1/1, ??:??/??:??] install: tugboat [26k]
running mktexlsr ...
done running mktexlsr.
\end{verbatim}

The other major thing \tinytex{} does is to emulate
\texttt{latexmk}, i.e., try to compile a \LaTeX{} document till all
cross-references are resolved. The reason to reinvent \texttt{latexmk}
in an R package is that \texttt{latexmk} cannot install missing packages
on-the-fly.

To sum it up, if R users compile a \LaTeX{} document via \tinytex{},
usually they will not need to know how many times they need to recompile
it, or run into errors due to missing packages. My implementation may be
clumsy, but the reaction from users seems to be positive anyway:
\url{github.com/yihui/tinytex/issues/7}. I hope this could give
some inspiration to developers in other communities, and I will be even
more excited if \TeX{} Live adds the native (and professional) support
someday, so I can drop my poor implementation.

\hypertarget{discussion}{%
\section{Discussion}\label{discussion}}

There is no free lunch. TinyTeX also has its drawbacks, and you have to
consider whether they matter to you. First of all, when installing
TinyTeX, you are always installing the very latest version of \TeX{} Live.
However, as I have mentioned, TinyTeX is a portable folder, so you can
save a copy of a certain version that worked for you, and use it in the
future.

Secondly, the installation of TinyTeX and the (automatic) installation
of additional \LaTeX{} packages requires an Internet connection. This may
be the biggest drawback of TinyTeX. If you plan to work offline, you
will have to make sure all packages have been installed in advance.

Thirdly, TinyTeX was created mainly for individual users who install
TinyTeX for themselves. If a sysadmin wants to install a shared copy of
TinyTeX for multiple users, there will be more technical details to
learn (in particular, issues related to permissions, the ``user mode'',
and packages that are not ``relocatable''). I have mentioned them on the
\acro{FAQ} page: \url{yihui.name/tinytex/faq/}.

Lastly, TinyTeX is essentially a version of \TeX{} Live installed through
an installation script. I did not provide prebuilt binaries, even though
it would be easy technically. I do not fully understand the \TeX{} Live
license and \LaTeX{} package licenses, but I guess I would be very likely
to violate these licenses if I provide binaries without also shipping
the source files inside at the same time. Anyway, installing TinyTeX
over the Internet usually takes only a minute or two, so this may not be
a big concern.

I hope you might find TinyTeX (and the R package \tinytex{}, if
you happen to be an R user, too) useful. If you have any feedback or
questions or bug reports, please feel free to post them to the Github
repository: \url{github.com/yihui/tinytex}.

\bibliographystyle{tugboat}
\bibliography{tinytex.bib}

\makesignature
\end{document}
